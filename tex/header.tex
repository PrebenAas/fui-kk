\documentclass[norsk,a4paper,11pt]{report}
\usepackage[T1]{fontenc}
\usepackage[utf8]{inputenc}
\usepackage{helvet}
\usepackage{babel,epsfig,moreverb,ifikompendiumforside,float}
\usepackage[
    colorlinks,
    citecolor=black,              % I like links with standard black color
    filecolor=black,
    linkcolor=black,
    urlcolor=black
]{hyperref}                       % Links in TOC etc.
\usepackage[all]{hypcap}          % Better links to floating environment

\author{Fagutvalget ved Institutt for informatikk}
\title{Kursevaluering}
\subtitle{Vår 2016}

\date{\today}

\floatstyle{ruled}
\restylefloat{figure}

% for paragrafer uten innrykk og tomme linjer mellom dem:
\setlength{\parindent}{0pt}
\setlength{\parskip}{1ex plus 0.5ex minus 0.2ex}

% Lagt inn av josek 19. okt 2009, for bedre utnyttelse av papirarealet:
\setlength{\textwidth}		{15.0cm}
\setlength{\textheight}		{24.5cm}

\begin{document}

\ififorside{}

% Lagt inn av josek 19. okt 2009, disse er best å ha etter uiosloforside:
\setlength{\topmargin}		{-1.0cm}
\setlength{\headsep}		{0.5cm}
\setlength{\oddsidemargin}	{0.5cm}
\setlength{\evensidemargin}	{0.5cm}
\setlength{\footskip}		{1.0cm}
\renewcommand{\thesection}{}
\renewcommand{\thesubsection}{}
\tableofcontents
\newpage
\chapter{Introduksjon}
\newpage
\chapter{Kurs}
\newpage
